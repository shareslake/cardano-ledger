\hypersetup{
  pdftitle={Formal Specification of the Cardano Ledger with Plutus
  Integration},
  breaklinks=true,
  bookmarks=true,
  colorlinks=false,
  linkcolor={blue},
  citecolor={blue},
  urlcolor={blue},
  linkbordercolor={white},
  citebordercolor={white},
  urlbordercolor={white}
}


\floatstyle{boxed}
\restylefloat{figure}
\cleardoublepage
\renewcommand{\thepage}{\arabic{page}}
\setcounter{page}{1}

\title{Formal Specification of the Cardano Ledger for the Babbage era}

\author{
   Andre Knispel \\ {\small \texttt{andre.knispel@iohk.io}} \\
   Jared Corduan \\ {\small \texttt{jared.corduan@iohk.io}} \\
   }

\date{}

\maketitle

\begin{abstract}
This document presents the modifications to the Alonzo ledger specification~
(see \cite{alonzo_spec}) for the Babbage era.
The Babbage era introduces two main groups of changes.

The first group involves new ways of providing data to Plutus scripts.
In particular, there is now support for
\href{https://cips.cardano.org/cips/cip31/}{reference inputs},
\href{https://cips.cardano.org/cips/cip32/}{inline datums}, and
\href{https://cips.cardano.org/cips/cip33/}{reference scripts}.
Additionally, the Babbage era supports
\href{https://github.com/cardano-foundation/CIPs/pull/216}{collateral outputs},
which reduces the risk of using collateral inputs for hardware wallets.

The second group of changes involves the handling of block headers.
We introduce a performance optimization, namely using a single VRF value for both
the leader check and the epoch nonce contribution.
We also remove the features introduced in the Shelley era which existed in order
to smoothly transition from a federated environment into a decentralized environment
(with respect to block production).
In particular, there is no longer an overlay schedule or a mechanism for adding
extra entropy to the epoch nonce.
\end{abstract}

\section*{List of Contributors}
\label{acknowledgements}

Alex Byaly,
Nicholas Clarke,
Duncan Coutts,
Sebastien Guillemot,
Philipp Kant,
Jan Mazak,
Michal Peyton Jones,
Tim Sheard,
Polina Vinogradova
